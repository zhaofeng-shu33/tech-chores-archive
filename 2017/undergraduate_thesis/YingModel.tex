\documentclass[12pt]{article}
\usepackage{xeCJK}%preamble part
\usepackage{graphicx}
\usepackage{indentfirst}
\usepackage[a4paper, inner=1.5cm, outer=3cm, top=2cm, bottom=3cm, bindingoffset=1cm]{geometry}
\usepackage{epstopdf}
\usepackage{listings}
\usepackage{array}
\usepackage{soul}
\usepackage{fontspec}
\usepackage{bm}
\usepackage{gensymb}
\usepackage{todonotes}
\usepackage{amsmath, amsthm, amssymb}
\usepackage[citecolor=blue]{hyperref}
\newtheorem{definition}{Definition}
\newtheorem{thm}{Theorem}[section]
\newtheorem{cor}[thm]{Corollary}
\newtheorem{lem}[thm]{Lemma}
\DeclareMathOperator{\sgn}{sgn}
\theoremstyle{remark}
\newtheorem*{rem}{Remark}
\usepackage{makecell}
\usepackage[lofdepth,lotdepth]{subfig}




\setlength{\extrarowheight}{4pt}
\setlength{\parindent}{1cm}
\begin{document}
\title{\textbf{\fontsize{15.75pt}{\baselineskip}{讨论}}} 

\author{\fontsize{12pt}{\baselineskip}{数33 赵丰}}
\maketitle
\large
行人从数轴上的0出发,地图是已知L处有一堵墙,行人不能
越过墙,假设行人每一步的步长$S_k \approx U[a,b],a>0$
行人前进用离散的随机过程来描述,我们利用均值对行人第k步的位置进行预测,想研究的问题是如果知道L处有墙,那么在行人实际撞墙前对其位置的预测误差能提高多少。

设行人前进k步的位置为$X_K,X_K=S_1+S_2+...+S_K$,$S_k$独立同分布,所以$X_K$的特征函数可以写成$S_i$特征函数乘积的形式。

$S_i$的特征函数为:
\begin{equation}
\phi_i(t)=\frac{e^{itb}-e^{ita}}{it(b-a)}
\end{equation}
$X_k$的特征函数为
\begin{equation}
\phi(t)=(\frac{e^{itb}-e^{ita}}{it(b-a)})^K
\end{equation}
再对特征函数做逆变换可得$X_k$的概率密度函数(\hl{how to derive?}):
\begin{equation}
P_k(x)=\frac{1}{(K-1)!(b-a)^K}\sum_{i=0}^{\tilde{K}(K,x)}(-1)^i C^i_K (x-Ka-i(b-a))^{K-1},Ka \leq x \leq Kb
\end{equation}
其中$\tilde{K}(K,x)$表示不大于$\frac{x-Ka}{b-a}$最大的整数。

这个K个均匀分布的随机变量的独立和服从Irwin–Hall 分布(\url{https://en.wikipedia.org/wiki/Irwin–Hall\_distribution}),是$Kb \leq L$的情形.
下面考虑$(K-1)b\leq L <Kb$,即第K步时有可能撞墙。

在有地图的情况下,对行人位置估计要比没有地图时要减小一点,相应的误差也要减小一点。

有地图情况下第K步分布的概率密度函数为:
\begin{equation}
P^*_K(x)=\frac{P_K(x)}{\int_0^L P_K(x)dx}
\end{equation}
$P^*_K(x)$形式比较复杂:
赵涵颖在$P_K(x)$的基础上尝试推导$P^*_{K+1}(x)$的表达式。她假设第K步还没有撞墙,而且$P^K(x)$可用均值为$(a+b)K/2$,方差为$(b-a)^2K/12$的正态分布密度函数进行近似。
赵涵颖认为$P^*_{K+1}$的密度函数比较复杂,不容易写出行人位置的参数估计和方差估计。

\end{document}