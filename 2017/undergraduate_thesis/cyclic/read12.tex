\documentclass[12pt]{article}
\usepackage{bm}
\usepackage{amsmath}
\begin{document}
\section{Conclusion}
the FIM of the ending point with loop has the following form:
\[
\bm{J}_n=\lambda\bm{I}_2+T_1\bm{u}_{n-1}\bm{u}_{n-1}^T+T_n\bm{u}_n\bm{u}_n^T+q(\bm{u}_{n-1}\bm{u}_{n}^T+\bm{u}_{n}\bm{u}_{n-1}^T)
\]
$T_1$ and $T_n$ have the continous fraction form and converges exponentially to some fixed value less than 1;
$q$ has continous product form show in this article and converges to zero exponentially:
\begin{equation}
q\leq \left(\cfrac{1}{1+\cfrac{\sqrt{\lambda^2+4\lambda}+\lambda}{2}}\right)^{n-1}
\end{equation}
\section{Content}
We consider the cyclic decomposition of special matrix (FIM) as follows:
\begin{equation}
\bm{J}=\begin{pmatrix}
                 \bm{B}_1 & \bm{A}_2 & \bm{0} & \dots & \bm{A}_1 \\
                 \bm{A}_2 & \bm{B}_2 & \bm{A}_3 & \dots & \bm{0} \\
                 \vdots & \vdots & \vdots & \ddots & \vdots \\
                 \bm{A}_1 & \dots & \bm{0} & \bm{A}_{n} & \bm{B}_{n}
               \end{pmatrix}.
\end{equation}
We still consider 
\begin{equation}\label{eq:t}
\bm{e}_{n}^{T}\bm{J}^{-1}\bm{e}_{n}
\end{equation}
We have already known the above 2 times 2 matrix equals $U_{n}^{-1}$ when $\bm{A}_1$ vanishes.
Now we will derive the general algebric form of (\ref{eq:t}) and compare it with the above special case.
We will also derive the closed from for $n\to \infty$ and each angle tends to zero, which corresponds to a closed circle
path with infinite small time sampling interval.

First we UL decompose $\bm{I}$ when $\bm{A}_1$ vanishes as follows:
\begin{equation}\label{eq:UL}
  \bm{J}=\begin{pmatrix}
                 \bm{I}_2 & \bm{L'}_2 & \bm{0} & \dots & \bm{0} \\
                 \bm{0} & \bm{I}_2 & \bm{0} & \dots & \bm{0} \\
                 \vdots & \vdots & \vdots & \ddots & \bm{L'}_{n} \\
                 \bm{0} & \dots & \bm{0} & \dots & \bm{I}_{2}
               \end{pmatrix}
               \begin{pmatrix}
                 \bm{U'}_1 & \bm{0} & \bm{0} & \dots & \bm{0} \\
                 \bm{A}_2 & \bm{U'}_2 & \bm{0} & \dots & \bm{0} \\
                 \vdots & \bm{A}_3 & \vdots & \ddots & \vdots \\
                 \bm{0} & \dots & \bm{0} & \bm{A}_n & \bm{U'}_{n}
               \end{pmatrix}.
\end{equation}
where $\bm{U'}_i$ satisfies:
\begin{equation}
\begin{cases}
  \bm{U'}_n &= \bm{B}_1 \\
  \bm{U'}_i &= \bm{B}_i-\bm{A}_{i+1}\bm{U}_{i+1}^{\bm{'}-1}\bm{A}_{i+1},i< n.
\end{cases}
\end{equation}
This form is dual to what we have derived in thesis. However, to make it appliable in this article, we should also have some knowledge on $\bm{L'}_i$. Therefore we rewrite the above equation as follows:
\begin{equation}
\begin{cases}
  \bm{L'}_i &= \bm{A}_i\bm{U}_i^{\bm{'}-1} \\
  \bm{U'}_i &= \bm{B}_i-\bm{L'}_{i+1}\bm{A}_{i+1}
\end{cases},i< n.
\end{equation}
We add prime to each UL to distinguish it from canonical LU form.

When $\bm{A}_1 \neq \bm{0}$, we first extract the (n-1) times (n-1) upper diagnal block from $\bm{J}$,denoted as $\bar{\bm{A}}$,
$\alpha=\begin{pmatrix}
\bm{A}_1\\ \vdots\\ \bm{A}_n
\end{pmatrix}
$,and $\bm{\bar{x}}$ is the first n-1 elements of $\bm{x}$,
then the equation \[
\bm{J}\bm{x}=\bm{e}_{n}
\]
can be expanded in block form:
\begin{equation}
\begin{pmatrix}
\bar{\bm{A}}&\bm{\alpha}\\
\bm{\alpha}^T&\bm{B}_n
\end{pmatrix}\begin{pmatrix}
\bm{\bar{x}}\\ \bm{x}_n
\end{pmatrix}=\begin{pmatrix}
\bm{0}\\ \bm{I}_2
\end{pmatrix}
\end{equation}
Therefore if $\bm{x}_n$ is known, we can solve 
\[
\bm{\bar{A}}\bm{\bar{x}}=-\bm{\alpha} \bm{x}_n
\]
We can find solution to $\bm{\bar{A}}\bm{\bar{\xi}}=\bm{\alpha}$,then $\bm{x}_1=-\bm{\xi}_1 \bm{x}_n$,
$\bm{x}_{n-1}=-\bm{\xi}_{n-1}\bm{x}_n$.
Taking $\bm{x}_1,\bm{x}_{n-1}$into the following equation we can solve $\bm{x}_n$ out:
\[
\bm{A}_1 \bm{x}_1+\bm{A}_{n}\bm{x}_{n-1}+\bm{B}_n \bm{x}_n=\bm{I}_2.
\]
It follows that:
\[
\bm{x}_n=(\bm{B}_n-\bm{A}_1\bm{\xi}_1-\bm{A}_n\bm{\xi}_{n-1})^{-1}
\]
$\bm{B}_n-\bm{A}_1\bm{\xi}_1-\bm{A}_n\bm{\xi}_{n-1}$ is the FIM of one node.

The first observation is that $A_1$introduces a non-trival term in the FIM with loop topology.

Then we investigate the structure of $\bm{\xi}_1$ and $\bm{\xi}_{n-1}$.

$\bm{\xi}_{n-1}$ can be solved with LU decomposition of $\bm{\bar{A}}$ as usual. However, the non-trival $\bm{A}_1$ in the first element of right hand side $\bm{\alpha}$ contributes to $\bm{\xi}_{n-1}$ non-trivally. 

First we solve the outer:
\begin{equation}
  \begin{pmatrix}
                 \bm{I}_2 & \bm{0} & \bm{0} & \dots & \bm{0} \\
                 \bm{L}_2 & \bm{I}_2 & \bm{0} & \dots & \bm{0} \\
                 \vdots & \vdots & \vdots & \ddots & \vdots \\
                 \bm{0} & \dots & \bm{0} & \bm{L}_{n-1} & \bm{I}_{2}
               \end{pmatrix}\begin{pmatrix}
                \bm{k}_1\\ \vdots \\ \bm{k}_{n-1}
               \end{pmatrix}=\begin{pmatrix}
\bm{A}_1 \\ \vdots \\ \bm{A}_n
\end{pmatrix}.
\end{equation}
It follows that:
\begin{equation}
\bm{k}_{n-1}=\bm{A}_n+(-1)^{n}\bm{L}_{n-1}\dots\bm{L}_3\bm{L}_2\bm{A}_1
\end{equation}
And the inner equation system:
\begin{equation}
\bm{\xi}_{n-1}=\bm{U}_{n-1}^{-1}\bm{k}_{n-1}
\end{equation}
The relationship between $\bm{L}_i$ and $\bm{U}_i$ is:
\begin{equation}
\bm{L}_i=\bm{A}_i\bm{U}^{-1}_{i-1}
\end{equation}
Therefore: 
\begin{equation}
-\bm{A}_n\bm{\xi}_{n-1}=-\bm{A}_n\bm{U}^{-1}_{n-1}\bm{A}_n+(-1)^{n-1}\bm{A}_n \prod_{i=n-1}^1 (\bm{U}_i^{-1}\bm{A}_i)
\end{equation}
The expression for $\bm{\xi}_1$ is similar, with UL decomposition instead.
\begin{equation}
  \begin{pmatrix}
                 \bm{I}_2 & \bm{L'}_2 & \bm{0} & \dots & \bm{0} \\
                 \bm{0} & \bm{I}_2 & \bm{0} & \dots & \bm{0} \\
                 \vdots & \vdots & \vdots & \ddots & \bm{L'}_{n} \\
                 \bm{0} & \dots & \bm{0} & \dots & \bm{I}_{2}
               \end{pmatrix}\begin{pmatrix}
                \bm{k'}_1\\ \vdots \\ \bm{k'}_{n-1}
               \end{pmatrix}=\begin{pmatrix}
\bm{A}_1 \\ \vdots \\ \bm{A}_n
\end{pmatrix}.
\end{equation}
it follows: 
\begin{equation}
\bm{k'}_{1}=\bm{A}_1+(-1)^n\bm{L}'_2\dots \bm{L}'_{n-1}\bm{A}_n
\end{equation}
And the inner equation system:
\begin{equation}
\xi_1=\bm{U}_1^{\bm{'}-1}\bm{k'}_1
\end{equation}
Therefore
\begin{equation}
-\bm{A}_1\bm{\xi}_1=-\bm{A}_1\bm{U}_1^{\bm{'}-1}\bm{A}_1+(-1)^{n-1}(\prod_{i=1}^{n-1}\bm{A}_i\bm{U}_i^{\bm{'}-1})\bm{A}_n
\end{equation}
FIM has the following form:
\begin{eqnarray*}
\bm{B}_n-\bm{A}_1\bm{\xi}_1-\bm{A}_n\bm{\xi}_{n-1}&=\\
\bm{B}_n-\bm{A}_n\bm{U}^{-1}_{n-1}\bm{A}_n-\bm{A}_1\bm{U}_1^{\bm{'}-1}\bm{A}_1&+
(-1)^{n-1}\bm{A}_n \prod_{i=n-1}^1 (\bm{U}_i^{-1}\bm{A}_i)+(-1)^{n-1}(\prod_{i=1}^{n-1}\bm{A}_i\bm{U}_i^{\bm{'}-1})\bm{A}_n
\end{eqnarray*}
$\bm{B}_n-\bm{A}_n\bm{U}^{-1}_{n-1}\bm{A}_n-\bm{A}_1\bm{U}_1^{\bm{'}-1}\bm{A}_1$ is the first order term which has continous fraction expansion and has been studied in detail in my paper.

In this article, we focus on the remaining continous product term.

%Firstly, we consider the limiting property, in the paper, we
%have deduced that 
%\begin{equation}
%\bm{U}_n=\bm{\hat{U}}_n \begin{pmatrix}
%T_1(i)&0\\0&\lambda
%\end{pmatrix}\bm{\hat{U}}_n^{-1}
%\end{equation}
%where 
%\begin{equation}
%\bm{\hat{U}}_n=\begin{pmatrix}
%\cos(\phi_n)&-\sin(\phi_n)\\
%\sin(\phi_n)&\cos(\phi_n)
%\end{pmatrix}
%\end{equation}
$\phi_n$ is the directional vector between the (n-1)th and n- th node.
We denote $\bm{A_i}=-\bm{u}_{i-1}\bm{u}_{i-1}^T$, and $\bm{u}_0=\bm{u}_n$ then
\begin{equation}
\bm{A}_n \prod_{i=n-1}^1 (\bm{U}_i^{-1}\bm{A}_i)=\left(\prod_{i=n-1}^{1} \bm{u}_{i}^T\bm{U}_i^{-1}\bm{u}_{i-1}\right)\bm{u}_n\bm{u}_1^T
\end{equation}
For each term $\bm{u}_i^T\bm{U}_i^{-1}\bm{u}_{i-1}$ (a real number), we have
\begin{align}
\bm{u}_i^T\bm{U}_i^{-1}\bm{u}_{i-1}=&\bm{u}_i^T(\lambda\bm{I}_2+k\bm{u}_{i-1}\bm{u}_{i-1}^T+\bm{u}_{i}\bm{u}_{i}^T)^{-1}\bm{u}_{i-1}\\
=&\frac{\lambda\cos(\Delta\phi)}{k \lambda+\lambda(1+\lambda)+k\sin^2(\Delta\phi)}
\end{align}
In the above formula,$\Delta\phi$ is the angle between $\bm{u}_{i-1}$ and $\bm{u}_{i}$, and
\[
k=1-\bm{u}_{i-1}^T\bm{U}_{i-1}^{-1}\bm{u}_{i-1}
\],
which is smaller than 1 and larger than zero.
Therefore, we get:
\begin{align}
\bm{u}_i^T\bm{U}_i^{-1}\bm{u}_{i+1}\leq&
\frac{\lambda}{k \lambda+\lambda(1+\lambda)}\\
=&\frac{1}{1+\lambda+k}
\end{align}
$k$ has continuous form expression, and its upper limit is(wrong!)
\begin{align}
k\leq & [1,\lambda,1,\lambda,\dots]\\
=&\frac{\sqrt{\lambda^2+4\lambda}-\lambda}{2}
\end{align}
Therefore
\begin{equation}
\bm{u}_i^T\bm{U}_i^{-1}\bm{u}_{i+1}\leq\cfrac{1}{1+\cfrac{\sqrt{\lambda^2+4\lambda}+\lambda}{2}}
\end{equation}•
We know that as n is very large,$\bm{B}_n-\bm{A}_n\bm{U}^{-1}_{n-1}\bm{A}_n-\bm{A}_1\bm{U}_1^{\bm{'}-1}\bm{A}_1$ is stable. 
And the product term decreases exponentially, the larger $\lambda$ is, the faster the decreasing speed is.

If the loop has many nodes, then the coupling term $\prod_{i=n-1}^1 (\bm{U}_i^{-1}\bm{A}_i)$ can be neglected.

Also it ccan be noticed for orthognal angle, $\cos(\Delta \phi)$ vanishes, therefore the coupling disapears.

Since $(-1)^n$ changes signs, we can treat $\prod_{i=n-1}^1 (\bm{U}_i^{-1}\bm{A}_i)$ as the oscillating term damping to zero exponentially.

In conclusion, for the FIM of the ending point, we have
\[
\bm{J}_n=\lambda\bm{I}_2+T_1\bm{u}_{n-1}^T\bm{u}_{n-1}^T+T_n\bm{u}_n^T\bm{u}_n^T+q(\bm{u}_{n-1}^T\bm{u}_{n}^T+\bm{u}_{n}^T\bm{u}_{n-1}^T)
\]
$T_1$ and $T_n$ have the continous fraction form and converges exponentially to some fixed value less than 1;
$q$ has continous product form show in this article and converges to zero exponentially.
%However, $\left(\prod_{i=1}^{n-1} \bm{u}_i^T\bm{U}_i^{-1}\bm{u}_{i+1}\right)$ may not stable even with non-zero $\lambda$. 
%
%We know that 
%\[
%T_1\leq \frac{\lambda+\sqrt{\lambda^2+4\lambda}}{2}
%\]
%If $\lambda<\frac{1}{2}$ and $\cos(\Delta \phi_i)$ is near 1, then it is possible that \[
%\bm{u}_i^T\bm{U}_i^{-1}\bm{u}_{i+1}>1
%\]
%As n is very large,$\left(\prod_{i=1}^{n-1} \bm{u}_i^T\bm{U}_i^{-1}\bm{u}_{i+1}\right)$,which is the coefficient of $\bm{u}_n\bm{u}_1^T$ dominates the 2 times 2 matrix.



\end{document}
