\documentclass[12pt]{article}
\usepackage{xeCJK}%preamble part
\usepackage{graphicx}
\usepackage{indentfirst}
\usepackage[a4paper, inner=1.5cm, outer=3cm, top=2cm, bottom=3cm, bindingoffset=1cm]{geometry}
\usepackage{epstopdf}
\usepackage{listings}
\usepackage{array}
\usepackage{fontspec}
\usepackage{bm}
\usepackage{gensymb}
\usepackage{todonotes}
\usepackage{amsmath, amsthm, amssymb}
\usepackage[citecolor=blue]{hyperref}
\newtheorem{definition}{Definition}
\newtheorem{thm}{Theorem}[section]
\newtheorem{cor}[thm]{Corollary}
\newtheorem{lem}[thm]{Lemma}
\DeclareMathOperator{\sgn}{sgn}
\theoremstyle{remark}
\newtheorem*{rem}{Remark}
\usepackage{makecell}
\usepackage[lofdepth,lotdepth]{subfig}




\setlength{\extrarowheight}{4pt}
\setlength{\parindent}{1cm}
\begin{document}
\title{\textbf{\fontsize{15.75pt}{\baselineskip}{讨论}}} 

\author{\fontsize{12pt}{\baselineskip}{数33 赵丰}}
\maketitle
\large
文献阅读结果:
\begin{thm}
设$n\le m,\alpha_i$是m维列向量,
$U_0=(\alpha_1,...,\alpha_n)$是$m\times n$的矩阵,则有:
\[
\text{dim}(U_0U_0^T)_{m\times m}=\text{dim}(U_0^TU_0)_{n\times n}=n
\]
\end{thm}
因为当讨论特殊网络中,直接处理FIM矩阵会非常困难,因此我尝试变换思路,讨论某种无偏定位算法来获得对定位误差的洞见。
本着由简单到复杂的思路,先讨论场景中只有N个锚点的情况,对于移动节点的估计量$(\hat{x},\hat{y})$适合下面的方程组:
\[
(\hat{x}-x_i)^2+(\hat{y}-x_i)^2=u_i^2,i=1,2,...N
\]
其中$u_i=\sqrt{(x-x_i)^2+(y-y_i)^2}+\text{noise},(x,y)$是节点的真实位置
上面的方程组超定且非线性,但可通过相减得到N-1个关于$(x,y)$的线性
方程,由于有对于噪声项方差已知且全部相等的假设,取哪一项作为被减项无关紧要,不妨取关于$(x_1,y_1)$的方程作为被减项,可得到:
\[
2(x_i-x_1)\hat{x}+2(y_i-y_1)\hat{y}=u_1^2-u_i^2+x_i^2-x_1^2+y_i^2-y_1^2,i=2,...N
\]
由于我们假设$N\geq 3$,上面的方程至少是有唯一解的,设
\[
A=2\left(
\begin{array}{cc}
x_2-x_1&y_2-y_1\\
\vdots&\vdots\\
x_n-x_1&y_n-y_1\\
\end{array}
\right),b=\left(
\begin{array}{c}
u_1^2-u_2^2+x_2^2-x_1^2+y_2^2-y_1^2\\
\vdots\\
u_1^2-u_n^2+x_n^2-x_1^2+y_n^2-y_1^2\\
\end{array}
\right)
\]
最小二乘求解$A\binom{\hat{x}}{\hat{y}}=b$的结果为:
$\bm{\hat{p}}=(A^TA)^{-1}A^Tb$
我们注意到只有$u_i$是高斯型随机变量,直接计算可知$\mathbb{E}(u_i^2)=(x-x_i)^2+(y-y_i)^2+\sigma^2$
所以$\mathbb{E}(b)=A\bm{p}$
\begin{thm}
$\bm{\hat{p}}=(A^TA)^{-1}A^Tb$是关于未知节点位置估计的无偏估计
\end{thm}
$\bm{\hat{p}}$均值为$\bm{p}$,下面我们求估计量$\bm{\hat{p}}$的协方差矩阵。
\begin{thm}
设X是一个n维随机向量,A是$m\times n$的矩阵,$Y=AX$,那么$\text{Cov}(Y)=A\text{Cov}(X)A^T$,Cov表示随机向量的协方差矩阵。
\end{thm}
\begin{proof}
因为$Y=AX$,有
\[
\begin{cases}
Y_i&=\sum_{k=1}^n a_{ik}X_k\\
Y_j&=\sum_{t=1}^n a_{jt}X_t\\
\end{cases}
\]
$1\leq i\leq j\leq m$
Cov(Y)的第i行第j列为$\text{Cov}(Y_i,Y_j)$
由Cov的线性性质,得
\[
\text{Cov}(Y_i,Y_j)=\sum_{k=1}^n\sum_{t=1}^n a_{ik}a_{jt}\text{Cov}(X_k,X_t)
\]
另一方面矩阵$A\text{Cov}(X)A^T$的第i行第j列为
\[
\sum_{t=1}^n(A\text{Cov}(X))_{it}a^t_{tj}
=\sum_{t=1}^n(\sum_{k=1}a_{ik}\text{Cov}(X_k,X_t))_{it}a_{jt}
\]
所以原命题成立。
\end{proof}
由上面的定理,我们只需先求Cov(b)
\begin{thm}
\[
Cov(b)=\xi_1 \bm{e}\bm{e}^T+\text{diag}\{\xi_2,...\xi_n\}
\],其中
\[
\xi_i=\mathbb{D}(u_i^2)=\mathbb{E}(u_i^4)-(\mathbb{E}(u_i^2))^2=4\sigma^2((x-x_i)^2+(y-y_i)^2)+2\sigma^4
\]
\end{thm}
\begin{proof}
取$b_i$和$b_j$作协方差,分$i\neq j$和$i = j$两种情况讨论,
首先可将$b_i$写成如下形式:
\[
b_i=u_1^2-u_i^2+\mathbb{E}(u_i^2)-\mathbb{E}(u_1^2)
\]
$i\neq j$即协方差阵的非对角元化简可得$\xi_1$,$i=j$即对角元为$\xi_1+\xi_i$,为求$\xi_i$,考虑求正态分布的四阶原点矩。
设$X\sim N(\mu,\sigma)$,则
\[
\mathbb{E}(X^4)=\mathbb{E}(X-\mu)^4+6\mu^2\mathbb{E}(X-\mu)^2+\mu^4
\]
通过Gamma函数可以求出$\mathbb{E}(\frac{X-\mu}{\sigma})^4=3$
所以
\[
\mathbb{E}(X^4)=3\sigma^4+6\sigma^2\mu^2+\mu^4
\]
从而定理得证。
\end{proof}
\end{document}