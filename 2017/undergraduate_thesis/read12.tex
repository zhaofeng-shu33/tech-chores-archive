\documentclass[12pt]{article}
\usepackage{xeCJK}%preamble part
\usepackage{graphicx}
\usepackage{indentfirst}
\usepackage[a4paper, inner=1.5cm, outer=3cm, top=2cm, bottom=3cm, bindingoffset=1cm]{geometry}
\usepackage{epstopdf}
\usepackage{listings}
\usepackage{array}
\usepackage{fontspec}
\usepackage{bm}
\usepackage{gensymb}
\usepackage{todonotes}
\usepackage{amsmath, amsthm, amssymb}
\usepackage[citecolor=blue]{hyperref}
\newtheorem{definition}{Definition}
\newtheorem{theorem}{Theorem}[section]
\newtheorem{cor}[theorem]{Corollary}
\newtheorem{lem}[theorem]{Lemma}
\DeclareMathOperator{\sgn}{sgn}
\theoremstyle{remark}
\newtheorem*{rem}{Remark}
\DeclareMathOperator{\K}{K}
\usepackage{makecell}
\usepackage[lofdepth,lotdepth]{subfig}




\setlength{\extrarowheight}{4pt}
\setlength{\parindent}{1cm}
\begin{document}
\title{\textbf{\fontsize{15.75pt}{\baselineskip}{讨论}}}

\author{\fontsize{12pt}{\baselineskip}{数33 赵丰}}
\maketitle
\large
\begin{equation}\label{eq:4_characteristic_polynomial}
\begin{split}
P(\lambda)= &(\lambda-a_1)(\lambda-a_2)(\lambda-a_3)(\lambda-a_4)(1+\epsilon(\frac{\cos^2(\theta)}{\lambda-a_1}+\\
 &\frac{\sin^2(\theta)}{\lambda-a_2}+\frac{\cos^2(\phi)}{\lambda-a_3}+\frac{\sin^2(\phi)}{\lambda-a_4}))
\end{split}
\end{equation}
\begin{equation}\label{eq:recursive_efim}
\begin{split}
T_{i-1}(N_a)& =\lambda+\frac{1}{1+\frac{\sin^2\theta_i}{\lambda}+\frac{\cos^2\theta_i}{T_i}},2\leq i\leq N_a-1\\
T_{N_a-1}(N_a)& = \lambda+\frac{1}{1+1/\lambda}
\end{split}
\end{equation}
其中$\theta_i=\angle <\bm{p}(t_i)-\bm{p}(t_{i-1}),\bm{p}(t_i)-\bm{p}(t_{i+1})>$
\begin{equation}\label{eq:starting_or_ending}
M^*=\lambda+\cfrac{1}{1+\cfrac{1}{\lambda+\cfrac{1}{1+\cfrac{1}{\lambda+\dots}}}}
=\frac{\lambda+\sqrt{4\lambda+\lambda^2}}{2}
\end{equation}

\begin{theorem}\label{theorem:arbitrary_curve}
若平面轨迹曲线参数化形式为$t\in[0,1]\rightarrow \bm{p}(t)=(x(t),y(t))$,满足
\begin{itemize}
\item $\bm{p}(t_1)\neq \bm{p}(t_2),\forall t_1\neq t_2$
\item $\bm{p}'(t)$存在且连续
\end{itemize}
对[0,1]区间有分割$0=t_1<t_2<\dots<t_{N_a-1}<t_{N_a}=1$,
记$\Delta t=\max_{1\leq i\leq N_a-1}|t_{i+1}-t_i|$
则式(\ref{eq:recursive_efim})给出的$T_1(N_a)$满足:
\begin{equation}\label{eq:limiting_cf}
\lim_{\Delta t\to 0}T_1(N_a)=M^*
\end{equation}
\end{theorem}
\begin{proof}
式(\ref{eq:starting_or_ending})给出了式(\ref{eq:limiting_cf})右端是$\bm{p}(t)$为直线的情形。由于对于任意的平面曲线和角度序列$\{\theta_i\}$,$T_1(N_a)$是关于$N_a$的增函数且小于$\lambda+1$,因此式(\ref{eq:limiting_cf})左端的极限总是存在的。
考虑由$\bm{p}(t)$确定的角度序列$\{\theta_i\}$以如下的方式趋近于直线对应的直线序列:
\[
\{\theta_1,\theta_2,\theta_3,\dots\}\rightarrow\{0,\theta_2,\theta_3,\dots\}\rightarrow
\{0,0,\theta_3,\dots\}\rightarrow\dots
\]
记将前n个角度置零后由式(\ref{eq:recursive_efim})确定的连分式为$K_n$,我们首先给出:
\begin{equation}\label{eq:arbitrary_curve_1}
\lim_{n\to\infty}K_n=\frac{\lambda+\sqrt{4\lambda+\lambda^2}}{2}
\end{equation}
为证式(\ref{eq:arbitrary_curve_1}),记角度序列$\{0,0,\dots,\theta_{n+1},\dots,\}$去掉前r项后对应的连分式为$K^r_n$
\[
|K_n-M^*|=\frac{|\frac{1}{M^*}-\frac{1}{K^2_n}|}{(1+\frac{1}{M^*})(1+\frac{1}{K^2_n})}\leq |\frac{1}{M^*}-\frac{1}{K^2_n}|
\]
\[
|\frac{1}{M^*}-\frac{1}{K^2_n}|= \frac{|\frac{1}{M^*}-\frac{1}{K^3_n}|}{K^2_n(M^*+1)(1+\frac{1}{K^3_n})}=
 \frac{|\frac{1}{M^*}-\frac{1}{K^3_n}|}{(M^*+1)(\lambda+1+\frac{1}{K^3_n+1})}\leq \frac{|\frac{1}{M^*}-\frac{1}{K^3_n}|}{(\lambda+1)^2}
\]
当$r<n$ 时,
\[
|\frac{1}{M^*}-\frac{1}{K^r_n}|\leq \frac{|\frac{1}{M^*}-\frac{1}{K^{r+1}_n}|}{(\lambda+1)^2}
\]
因此:
\[
|K_n-M^*|\leq \frac{|\frac{1}{M^*}-\frac{1}{K^{n}_n}|}{(\lambda+1)^{2(n-2)}}
\]
故式(\ref{eq:arbitrary_curve_1})成立。
补充$K_0=\lim_{\Delta t\to 0}T_1(N_a)$这样式(\ref{eq:limiting_cf})即等价为
\begin{equation}\label{eq:equivalent_limiting_cf}
\sum_{i=1}^{\infty}(K_{i-1}-K_{i})=0
\end{equation}
先考虑$K_0-K_1$,二者的差别是$\theta_1$是否为0,
\[
K_0-K_1=\frac{1}{1+\frac{1}{K_1^1}}-\frac{1}{1+\frac{1}{K_1^1}+\sin^2\theta_1(\frac{1}{\lambda}-\frac{1}{K_1^1})}\leq
\sin^2\theta_1(\frac{1}{\lambda}-\frac{1}{K_1^1})
\]
类似式(\ref{eq:arbitrary_curve_1})的推导:
\[
K_r-K_{r+1}\leq \sin^2\theta_{r+1}(\frac{1}{\lambda}-\frac{1}{K_{r+1}^{r+1}})\frac{1}{(\lambda+1)^{2r}}
\]
\end{proof}
由条件$\bm{p}'(t)$存在且连续可得切向量是连续变化的,由微分中值定理在闭区间内存在常数c使得角度变化量$\theta_i\leq c\Delta t$
由正弦函数的单调性推出:
\[
K_r-K_{r+1}\leq \sin^2 (c\Delta t) \frac{1}{\lambda}\frac{1}{(\lambda+1)^{2r}}
\]
因此\[
0\leq \sum_{i=1}^{N_a}(K_{i-1}-K_{i})\leq \sin^2 (c\Delta t) \frac{1}{\lambda}\sum_{i=1}^{\infty}\frac{1}{(\lambda+1)^{2i}}
\]
无穷级数$\sum_{i=1}^{\infty}\frac{1}{(\lambda+1)^{2i}}$收敛,所以当$\Delta t\to 0$时$N_a\to \infty$,式(\ref{eq:equivalent_limiting_cf})成立。
\end{document}
