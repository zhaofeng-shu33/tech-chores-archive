\documentclass[12pt]{article}
\usepackage{xeCJK}%preamble part
\usepackage{graphicx}
\usepackage{indentfirst}
\usepackage[a4paper, inner=1.5cm, outer=3cm, top=2cm, bottom=3cm, bindingoffset=1cm]{geometry}
\usepackage{epstopdf}
\usepackage{listings}
\usepackage{array}
\usepackage{fontspec}
\usepackage{gensymb}
\usepackage{todonotes}
\usepackage{amsmath}
\usepackage[citecolor=blue]{hyperref}

\usepackage{makecell}
\usepackage[lofdepth,lotdepth]{subfig}




\setlength{\extrarowheight}{4pt}
\setlength{\parindent}{1cm}
\begin{document}
\title{\textbf{\fontsize{15.75pt}{\baselineskip}{讨论}}} 

\author{\fontsize{12pt}{\baselineskip}{数33 赵丰}}
\maketitle
\large
add some new line here
在协作定位中,关于$N_a$个移动节点的FIM是一个对称的主对角占优矩阵,由这种矩阵的性质可以得到FIM是半正定的。
关于矩阵求逆在做差的化简问题,我只能硬算,不知道有什么好的方法。
从J(0)-J(1)归纳出J(1)-J(2)的步骤:
\[
J(1)=(\Sigma_0+B_0B_0^H-B_0F_1^H
(\Sigma_1+F_1F_1^H)^{-1}F_1B_0^H)^{-1}
\]
设
\[
\tilde{\Sigma_1}=\Sigma_1+B_1B_1^H-B_1F_2^H
(\Sigma_2+F_2F_2^H)^{-1}F_2B_1^H
\]
则
\[
J(2)=(\Sigma_0+B_0B_0^H-B_0F_1^H
(\tilde{\Sigma_1}+F_1F_1^H)^{-1}F_1B_0^H)^{-1}
\]
推导J(1)-J(2)表达式如下:
\[
J(1)-J(2)=(J(0)-J(2))-(J(0)-J(1))
\]
由J(0)-J(1)的表达式的一般形式可得:
\[
=(\Sigma_0^{-1}B_0)((I+B_0^H\Sigma_0^{-1}B_0+F_1^H\tilde{\Sigma_1}^{-1}F_1)^{-1}-
\]
\[
(I+B_0^H\Sigma_0^{-1}B_0+F_1^H\Sigma_1^{-1}F_1)^{-1})B_0^H\Sigma_0^{-1}
\]
记$T=I+B_0^H\Sigma_0^{-1}B_0$
则有
\[
J(1)-J(2)=(\Sigma_0^{-1}B_0)((T+F_1^H\tilde{\Sigma_1}^{-1}F_1)^{-1}-(T+F_1^H\Sigma_1^{-1}F_1)^{-1})B_0^H\Sigma_0^{-1}
\]
下面化简
\[
(T+F_1^H\tilde{\Sigma_1}^{-1}F_1)^{-1}-(T+F_1^H\tilde{\Sigma_1}^{-1}F_1)^{-1}
\]
\[
=(T+F_1^H\Sigma_1^{-1}F_1)^{-1}((T+F_1^H\Sigma_1^{-1}F_1)(T+F_1^H\tilde{\Sigma_1}^{-1}F_1)^{-1}
(T+F_1^H\Sigma_1^{-1}F_1)-
\]
\[
(T+F_1^H\Sigma_1^{-1}F_1))(T+F_1^H\Sigma_1^{-1}F_1)^{-1}
\]
进一步化简
\[
(T+F_1^H\Sigma_1^{-1}F_1)(T+F_1^H\tilde{\Sigma_1}^{-1}F_1)^{-1}
(T+F_1^H\Sigma_1^{-1}F_1)-(T+F_1^H\Sigma_1^{-1}F_1)
\]
\[
=F_1^H(\Sigma_1^{-1}-\tilde{\Sigma_1}^{-1})F_1(T+F_1^H\tilde{\Sigma_1}^{-1}F_1)^{-1}(T+F_1^H\Sigma_1^{-1}F_1)
\]
\[
=F_1^H(\Sigma_1^{-1}-\tilde{\Sigma_1}^{-1})F_1(T+F_1^H\tilde{\Sigma_1}^{-1}F_1)^{-1}T(F_1^{-1}\Sigma_1F_1^{-H}+T^{-1})F_1^H\Sigma_1^{-1}F_1
\]
记$\Lambda=\Sigma_1+F_1T^{-1}F_1^H$,则
\[
=F_1^H(\Sigma_1^{-1}-\tilde{\Sigma_1}^{-1})F_1(T+F_1^H\tilde{\Sigma_1}^{-1}F_1)^{-1}TF_1^{-1}\Lambda\Sigma_1^{-1}F_1
\]
\[
=F_1^H\Sigma_1^{-1}(\tilde{\Sigma_1}-\Sigma_1)(\tilde{\Sigma_1}+F_1T^{-1}F_1^H))^{-1}\Lambda\Sigma_1^{-1}F_1
\]
设$\lambda=\tilde{\Sigma_1}-\Sigma_1$,则上式化为
\[
=F_1^H\Sigma_1^{-1}\Lambda(\Lambda^{-1}\lambda(\Lambda+\lambda)^{-1})\Lambda\Sigma_1^{-1}F_1
\]
\[
=F_1^H\Sigma_1^{-1}\Lambda(\Lambda^{-1}-(\Lambda+\lambda)^{-1})\Lambda\Sigma_1^{-1}F_1
\]
由J(1)-J(0)的表达式的一般形式,推出
\[
\Lambda^{-1}-(\Lambda+\lambda)^{-1}=\Lambda^{-1}B_1(I+B_1\Lambda^{-1}B_1^H+F_2^H\Sigma_2F_2)^{-1}B_1^H\Lambda^{-1}
\]
整理中间结果得到$J(1)-J(2)$为
\[
(\Sigma_0^{-1}B_0)(T+F_1^H\Sigma_1^{-1}F_1)^{-1}F_1^H\Sigma_1^{-1}B_1(I+B_1\Lambda^{-1}B_1^H+F_2^H\Sigma_2F_2)^{-1}B_1^H
\]
\[
\Sigma_1^{-1}F_1(T+F_1^H\Sigma_1^{-1}F_1)^{-1}B_0^H\Sigma_0^{-1}
\]
上述归纳具有一般性,第k层的FIM具有迭代的特点,可以通过反复使用EFIM数值求解,从而大幅降低了计算量。
\end{document}