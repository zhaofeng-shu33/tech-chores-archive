\documentclass[12pt]{article}
%DIF LATEXDIFF DIFFERENCE FILE
%DIF DEL read2.tex       Wed May 17 19:16:54 2017
%DIF ADD read2diff.tex   Wed May 17 20:04:34 2017
\usepackage{xeCJK}%preamble part
\usepackage{graphicx}
\usepackage{indentfirst}
\usepackage[a4paper, inner=1.5cm, outer=3cm, top=2cm, bottom=3cm, bindingoffset=1cm]{geometry}
\usepackage{epstopdf}
\usepackage{listings}
\usepackage{array}
\usepackage{fontspec}
\usepackage{gensymb}
\usepackage{todonotes}
\usepackage{amsmath}
\usepackage[citecolor=blue]{hyperref}

\usepackage{makecell}
\usepackage[lofdepth,lotdepth]{subfig}




\setlength{\extrarowheight}{4pt}
\setlength{\parindent}{1cm}
%DIF PREAMBLE EXTENSION ADDED BY LATEXDIFF
%DIF UNDERLINE PREAMBLE %DIF PREAMBLE
\RequirePackage[normalem]{ulem} %DIF PREAMBLE
\RequirePackage{color}\definecolor{RED}{rgb}{1,0,0}\definecolor{BLUE}{rgb}{0,0,1} %DIF PREAMBLE
\providecommand{\DIFaddtex}[1]{{\protect\color{blue}\uwave{#1}}} %DIF PREAMBLE
\providecommand{\DIFdeltex}[1]{{\protect\color{red}\sout{#1}}}                      %DIF PREAMBLE
%DIF SAFE PREAMBLE %DIF PREAMBLE
\providecommand{\DIFaddbegin}{} %DIF PREAMBLE
\providecommand{\DIFaddend}{} %DIF PREAMBLE
\providecommand{\DIFdelbegin}{} %DIF PREAMBLE
\providecommand{\DIFdelend}{} %DIF PREAMBLE
%DIF FLOATSAFE PREAMBLE %DIF PREAMBLE
\providecommand{\DIFaddFL}[1]{\DIFadd{#1}} %DIF PREAMBLE
\providecommand{\DIFdelFL}[1]{\DIFdel{#1}} %DIF PREAMBLE
\providecommand{\DIFaddbeginFL}{} %DIF PREAMBLE
\providecommand{\DIFaddendFL}{} %DIF PREAMBLE
\providecommand{\DIFdelbeginFL}{} %DIF PREAMBLE
\providecommand{\DIFdelendFL}{} %DIF PREAMBLE
%DIF HYPERREF PREAMBLE %DIF PREAMBLE
\providecommand{\DIFadd}[1]{\texorpdfstring{\DIFaddtex{#1}}{#1}} %DIF PREAMBLE
\providecommand{\DIFdel}[1]{\texorpdfstring{\DIFdeltex{#1}}{}} %DIF PREAMBLE
\newcommand{\DIFscaledelfig}{0.5}
%DIF HIGHLIGHTGRAPHICS PREAMBLE %DIF PREAMBLE
\RequirePackage{settobox} %DIF PREAMBLE
\RequirePackage{letltxmacro} %DIF PREAMBLE
\newsavebox{\DIFdelgraphicsbox} %DIF PREAMBLE
\newlength{\DIFdelgraphicswidth} %DIF PREAMBLE
\newlength{\DIFdelgraphicsheight} %DIF PREAMBLE
% store original definition of \includegraphics %DIF PREAMBLE
\LetLtxMacro{\DIFOincludegraphics}{\includegraphics} %DIF PREAMBLE
\newcommand{\DIFaddincludegraphics}[2][]{{\color{blue}\fbox{\DIFOincludegraphics[#1]{#2}}}} %DIF PREAMBLE
\newcommand{\DIFdelincludegraphics}[2][]{% %DIF PREAMBLE
\sbox{\DIFdelgraphicsbox}{\DIFOincludegraphics[#1]{#2}}% %DIF PREAMBLE
\settoboxwidth{\DIFdelgraphicswidth}{\DIFdelgraphicsbox} %DIF PREAMBLE
\settoboxtotalheight{\DIFdelgraphicsheight}{\DIFdelgraphicsbox} %DIF PREAMBLE
\scalebox{\DIFscaledelfig}{% %DIF PREAMBLE
\parbox[b]{\DIFdelgraphicswidth}{\usebox{\DIFdelgraphicsbox}\\[-\baselineskip] \rule{\DIFdelgraphicswidth}{0em}}\llap{\resizebox{\DIFdelgraphicswidth}{\DIFdelgraphicsheight}{% %DIF PREAMBLE
\setlength{\unitlength}{\DIFdelgraphicswidth}% %DIF PREAMBLE
\begin{picture}(1,1)% %DIF PREAMBLE
\thicklines\linethickness{2pt} %DIF PREAMBLE
{\color[rgb]{1,0,0}\put(0,0){\framebox(1,1){}}}% %DIF PREAMBLE
{\color[rgb]{1,0,0}\put(0,0){\line( 1,1){1}}}% %DIF PREAMBLE
{\color[rgb]{1,0,0}\put(0,1){\line(1,-1){1}}}% %DIF PREAMBLE
\end{picture}% %DIF PREAMBLE
}\hspace*{3pt}}} %DIF PREAMBLE
} %DIF PREAMBLE
\LetLtxMacro{\DIFOaddbegin}{\DIFaddbegin} %DIF PREAMBLE
\LetLtxMacro{\DIFOaddend}{\DIFaddend} %DIF PREAMBLE
\LetLtxMacro{\DIFOdelbegin}{\DIFdelbegin} %DIF PREAMBLE
\LetLtxMacro{\DIFOdelend}{\DIFdelend} %DIF PREAMBLE
\DeclareRobustCommand{\DIFaddbegin}{\DIFOaddbegin \let\includegraphics\DIFaddincludegraphics} %DIF PREAMBLE
\DeclareRobustCommand{\DIFaddend}{\DIFOaddend \let\includegraphics\DIFOincludegraphics} %DIF PREAMBLE
\DeclareRobustCommand{\DIFdelbegin}{\DIFOdelbegin \let\includegraphics\DIFdelincludegraphics} %DIF PREAMBLE
\DeclareRobustCommand{\DIFdelend}{\DIFOaddend \let\includegraphics\DIFOincludegraphics} %DIF PREAMBLE
\LetLtxMacro{\DIFOaddbeginFL}{\DIFaddbeginFL} %DIF PREAMBLE
\LetLtxMacro{\DIFOaddendFL}{\DIFaddendFL} %DIF PREAMBLE
\LetLtxMacro{\DIFOdelbeginFL}{\DIFdelbeginFL} %DIF PREAMBLE
\LetLtxMacro{\DIFOdelendFL}{\DIFdelendFL} %DIF PREAMBLE
\DeclareRobustCommand{\DIFaddbeginFL}{\DIFOaddbeginFL \let\includegraphics\DIFaddincludegraphics} %DIF PREAMBLE
\DeclareRobustCommand{\DIFaddendFL}{\DIFOaddendFL \let\includegraphics\DIFOincludegraphics} %DIF PREAMBLE
\DeclareRobustCommand{\DIFdelbeginFL}{\DIFOdelbeginFL \let\includegraphics\DIFdelincludegraphics} %DIF PREAMBLE
\DeclareRobustCommand{\DIFdelendFL}{\DIFOaddendFL \let\includegraphics\DIFOincludegraphics} %DIF PREAMBLE
%DIF END PREAMBLE EXTENSION ADDED BY LATEXDIFF

\begin{document}
\title{\textbf{\fontsize{15.75pt}{\baselineskip}{讨论}}} 

\author{\fontsize{12pt}{\baselineskip}{数33 赵丰}}
\maketitle
\large
\DIFaddbegin \DIFadd{add some new line here
}\DIFaddend 在协作定位中,关于$N_a$个移动节点的FIM是一个对称的主对角占优矩阵,由这种矩阵的性质可以得到FIM是半正定的。
关于矩阵求逆在做差的化简问题,我只能硬算,不知道有什么好的方法。
从J(0)-J(1)归纳出J(1)-J(2)的步骤:
\[
J(1)=(\Sigma_0+B_0B_0^H-B_0F_1^H
(\Sigma_1+F_1F_1^H)^{-1}F_1B_0^H)^{-1}
\]
设
\[
\tilde{\Sigma_1}=\Sigma_1+B_1B_1^H-B_1F_2^H
(\Sigma_2+F_2F_2^H)^{-1}F_2B_1^H
\]
则
\[
J(2)=(\Sigma_0+B_0B_0^H-B_0F_1^H
(\tilde{\Sigma_1}+F_1F_1^H)^{-1}F_1B_0^H)^{-1}
\]
推导J(1)-J(2)表达式如下:
\[
J(1)-J(2)=(J(0)-J(2))-(J(0)-J(1))
\]
由J(0)-J(1)的表达式的一般形式可得:
\[
=(\Sigma_0^{-1}B_0)((I+B_0^H\Sigma_0^{-1}B_0+F_1^H\tilde{\Sigma_1}^{-1}F_1)^{-1}-
\]
\[
(I+B_0^H\Sigma_0^{-1}B_0+F_1^H\Sigma_1^{-1}F_1)^{-1})B_0^H\Sigma_0^{-1}
\]
记$T=I+B_0^H\Sigma_0^{-1}B_0$
则有
\[
J(1)-J(2)=(\Sigma_0^{-1}B_0)((T+F_1^H\tilde{\Sigma_1}^{-1}F_1)^{-1}-(T+F_1^H\Sigma_1^{-1}F_1)^{-1})B_0^H\Sigma_0^{-1}
\]
下面化简
\[
(T+F_1^H\tilde{\Sigma_1}^{-1}F_1)^{-1}-(T+F_1^H\tilde{\Sigma_1}^{-1}F_1)^{-1}
\]
\[
=(T+F_1^H\Sigma_1^{-1}F_1)^{-1}((T+F_1^H\Sigma_1^{-1}F_1)(T+F_1^H\tilde{\Sigma_1}^{-1}F_1)^{-1}
(T+F_1^H\Sigma_1^{-1}F_1)-
\]
\[
(T+F_1^H\Sigma_1^{-1}F_1))(T+F_1^H\Sigma_1^{-1}F_1)^{-1}
\]
进一步化简
\[
(T+F_1^H\Sigma_1^{-1}F_1)(T+F_1^H\tilde{\Sigma_1}^{-1}F_1)^{-1}
(T+F_1^H\Sigma_1^{-1}F_1)-(T+F_1^H\Sigma_1^{-1}F_1)
\]
\[
=F_1^H(\Sigma_1^{-1}-\tilde{\Sigma_1}^{-1})F_1(T+F_1^H\tilde{\Sigma_1}^{-1}F_1)^{-1}(T+F_1^H\Sigma_1^{-1}F_1)
\]
\[
=F_1^H(\Sigma_1^{-1}-\tilde{\Sigma_1}^{-1})F_1(T+F_1^H\tilde{\Sigma_1}^{-1}F_1)^{-1}T(F_1^{-1}\Sigma_1F_1^{-H}+T^{-1})F_1^H\Sigma_1^{-1}F_1
\]
记$\Lambda=\Sigma_1+F_1T^{-1}F_1^H$,则
\[
=F_1^H(\Sigma_1^{-1}-\tilde{\Sigma_1}^{-1})F_1(T+F_1^H\tilde{\Sigma_1}^{-1}F_1)^{-1}TF_1^{-1}\Lambda\Sigma_1^{-1}F_1
\]
\[
=F_1^H\Sigma_1^{-1}(\tilde{\Sigma_1}-\Sigma_1)(\tilde{\Sigma_1}+F_1T^{-1}F_1^H))^{-1}\Lambda\Sigma_1^{-1}F_1
\]
设$\lambda=\tilde{\Sigma_1}-\Sigma_1$,则上式化为
\[
=F_1^H\Sigma_1^{-1}\Lambda(\Lambda^{-1}\lambda(\Lambda+\lambda)^{-1})\Lambda\Sigma_1^{-1}F_1
\]
\[
=F_1^H\Sigma_1^{-1}\Lambda(\Lambda^{-1}-(\Lambda+\lambda)^{-1})\Lambda\Sigma_1^{-1}F_1
\]
由J(1)-J(0)的表达式的一般形式,推出
\[
\Lambda^{-1}-(\Lambda+\lambda)^{-1}=\Lambda^{-1}B_1(I+B_1\Lambda^{-1}B_1^H+F_2^H\Sigma_2F_2)^{-1}B_1^H\Lambda^{-1}
\]
整理中间结果得到$J(1)-J(2)$为
\[
(\Sigma_0^{-1}B_0)(T+F_1^H\Sigma_1^{-1}F_1)^{-1}F_1^H\Sigma_1^{-1}B_1(I+B_1\Lambda^{-1}B_1^H+F_2^H\Sigma_2F_2)^{-1}B_1^H
\]
\[
\Sigma_1^{-1}F_1(T+F_1^H\Sigma_1^{-1}F_1)^{-1}B_0^H\Sigma_0^{-1}
\]
上述归纳具有一般性,第k层的FIM具有迭代的特点,可以通过反复使用EFIM数值求解,从而大幅降低了计算量。
\end{document}