\documentclass[12pt]{article}
\usepackage{xeCJK}%preamble part
\usepackage{graphicx}
\usepackage{indentfirst}
\usepackage[a4paper, inner=1.5cm, outer=3cm, top=2cm, bottom=3cm, bindingoffset=1cm]{geometry}
\usepackage{epstopdf}
\usepackage{listings}
\usepackage{array}
\usepackage{fontspec}
\usepackage{bm}
\usepackage{gensymb}
\usepackage{todonotes}
\usepackage{amsmath}
\usepackage[citecolor=blue]{hyperref}

\usepackage{makecell}
\usepackage[lofdepth,lotdepth]{subfig}




\setlength{\extrarowheight}{4pt}
\setlength{\parindent}{1cm}
\begin{document}
\title{\textbf{\fontsize{15.75pt}{\baselineskip}{讨论}}} 

\author{\fontsize{12pt}{\baselineskip}{数33 赵丰}}
\maketitle
\large
连长虽然指出杨思怡的文章虽然以最小跳数定义了分层结构,但她默认了
同层移动节点不协作且相对目标节点,分层实际上是一个tree structure,每个移动节点有且仅能属于某一层,否则FIM无法按层重排分解。这是一种近似方法,下面在这些假定下整理她文中的一些观点:


在之前的讨论中,有时假定$B_k,F_k$是最小单元即2乘2的矩阵,这次分析它们的精细结构,为简便假定Range Information Intensity均为1,
设第k层移动节点集合$V_k={p_1^k,p_2^k,...p_{n_k}^k}$,第k+1层移动节点集合$V_{k+1}$类似的定义。假设这两层节点不是所有的节点都能彼此通信,而且同层的节点不协作,来自第k层的移动节点i和来自第k+1层的移动节点j+1组成了一条通信链路$e^k_{i,j}$,上标k表明这条链路是第k+1层链路集合,仿此,第k+1层链路集合$\varepsilon$记为$\{
e^k_{i_1,j_1},e^k_{i_2,j_2},...e^k_{i_{m_k},j_{m_k}}\}$
集合的势$|\varepsilon|$小于$\frac{n_k n_{k+1}}{2}$,且假定至少有一条链路存在,
每一条链路对信息量的贡献是一个2乘2的矩阵$\bm{r}*\bm{r}^T$,这里$\bm{r}$是该链路的单位方向向量。两层节点协作的耦合矩阵$\bm{C}$是一个$n_k \times n_{k+1}$维的(最小单位是2乘2矩阵)。假定通信链路$e^k_{i,j}$存在,那么$\bm{C}$的第i行第j列的元素就是$\bm{r}*\bm{r}^T$。杨思怡的文章指出$\bm{C}$可以拆成维数是$n_k \times |\varepsilon|$的矩阵$\bm{B}_k$和$|\varepsilon| \times n_{k+1}$的矩阵$\bm{F}_{k+1}^T$,$\bm{B}_k$矩阵每一列对应一条通信链路,每一列只有1个非零元,且非零元所在的行数等于第k层节点的序号;同理$\bm{F}_{k+1}^T$矩阵每一行对应一条通信链路,每一行只有1个非零元,且非零元所在的列数等于第k+1层节点的序号。$\bm{B}_k,\bm{F}_{k+1}$具有好的性质是其各行相互正交。杨思怡还指出方阵$\bm{B}\bm{B}^T$
是对FIM中第k层节点对应的对角元的贡献,由于$\bm{B}_k$的正交性质(也就是同层节点不协作),使得方阵$\bm{B}\bm{B}^T$是一个对角阵,对角元为$\bm{B}$各行2范数的平方,当然,以上的讨论都是把2乘2的矩阵当成数来处理的,2范数的平方这种表述应该适当的修改。FIM中第k层节点对应的对角元还有来自第k-1层节点的协作贡献,分析思路是一样的。

基于如上分析,可以考虑去掉第k层链路集合$\varepsilon$某条链路对目标节点定位精度的影响。矩阵维数不变但$\bm{B}_k,\bm{F}_{k+1}$的列数减少1。
杨思怡的思路是沿用求J(k)的思路把某条链路的影响用代数表达式写出来,一方面她的附录证明太简洁我看不太懂她的式子是怎么推的,另一方面正如连长所说给出的表达式缺少启发性。我继续沿用之前的思路,在镜像杨思怡的矩阵基础上,可以把有没有$e^k_{i,j}$转化成在一个k步迭代运算中研究初值有误差矩阵$\bm{E}$的情况下对计算结果的最终影响:
初始矩阵$A_1$的误差项是$-(\bm{r}_{i,j}\bm{r}_{i,j}^T)\bm{e_j}\bm{e_j}^T$,这里$(\bm{r}_{i,j}\bm{r}_{i,j}^T)$应该看成$n_{k+1}$维的方阵第j行第j列的元素,$\bm{e_j}$是维数为$n_{k+1}$的单位向量,只在第j维取1。
$A_2$的误差项是$-(\bm{r}_{i,j}\bm{r}_{i,j}^T)\bm{e_i}\bm{e_i}^T$
$\bm{e_i}$是维数为$n_{k}$的单位向量。
$D_1$的误差项是$-(\bm{r}_{i,j}\bm{r}_{i,j}^T)\bm{e_i}\bm{e_j}^T$
而上次讨论(1)式可以从右下角很自然地拓展成k+1维矩阵,所以我们在接下来的工作中应重点研究$A_1,A_2,D_1$的误差在数列递推公式(非线性)中的误差传播问题。
递推公式中$A_k,D_k$和具体网络结构有关,我们下面先对棋盘网络给出$B_{k-1},
F_k$的表达式。
对于棋盘网络,选定一个移动节点为目标节点,第一层移动节点有4个,
第二层移动节点有3*4-4=8个,第k层移动节点有(k+1)*4-4=4k个。所以
只考虑前k层节点,FIM在层分块意义上是k+1维的,左上角的块是目标节点,是2*2的方阵,主对角线上依次是8*8的方阵,...8k*8k的方阵。
以下讨论以2*2的方阵为最小元,
我们考虑的$\bm{B}_{k},\bm{F}_{k}$的行数为4k。
对于每一层的节点,我们遵循从x轴正半轴逆时针转1圈的顺序从小到大给
节点排序,第一层链路集合为$\varepsilon_1=\{e^1_{1,1},e^1_{1,2},e^1_{1,3},e^1_{1,4}\},|\varepsilon_1|=4$第二层链路集为
$\varepsilon=\{e^2_{1,1},e^2_{1,2},e^2_{2,2},
e^2_{2,3},e^2_{2,4},e^2_{3,4},e^2_{3,5},
e^2_{3,6},e^2_{4,6},e^2_{4,7},e^2_{4,8},e^2_{1,8}\},|\varepsilon_2|=3\times 4,|\varepsilon_k|=3\times 4(k-1)$

对于棋盘阵,不同层的节点协作只产生两种类型的链路单位向量$\bm{i}=(1,0)^T$和$\bm{j}=(0,1)^T$,对于我们假定的
节点顺序$\bm{i}$和$\bm{j}$是交错出现的,
\[
B_0=(\bm{i},\bm{j},\bm{i},\bm{j})
\]
\[
F_1=\left(
\begin{array}{cccc}
\bm{i} & 0 & 0 & 0 \\
0 &\bm{j}& 0 & 0 \\
0 & 0 & \bm{i} & 0\\
0 & 0 & 0 & \bm{j}\\
\end{array}
\right),
\bm{I}_i:=\bm{i}\bm{i}^T=\left(
\begin{array}{cc}
1 & 0\\
0 & 0 \\
\end{array}
\right),
\bm{I}_j:=\bm{j}\bm{j}^T=\left(
\begin{array}{cc}
0 & 0\\
0 & 1 \\
\end{array}
\right)
\]
\[
\bm{i}\bm{j}^T=\left(
\begin{array}{cc}
0 & 1\\
0 & 0 \\
\end{array}
\right),
\bm{j}\bm{i}^T=\left(
\begin{array}{cc}
0 & 0\\
1 & 0 \\
\end{array}
\right),B_1=\left(
\begin{array}{cccc}
\bm{i}\: \bm{j} &  &  & \bm{j}  \\
 &\bm{i}\: \bm{j} \: \bm{i}&  &  \\
 &  & \bm{j}\: \bm{i} \: \bm{j} & \\
 &  &  & \bm{i}\: \bm{j} \: \bm{i}\\
\end{array}
\right)_{4 \times 12}\]
\[
F_2=\left(
\begin{array}{cccccccc}
\bm{i}&&&&&&&  \\
&\bm{j} \: \bm{i}&&&&&&\\
&&\bm{j}&&&&&\\
&&&\bm{i}\: \bm{j}&&&&\\
&&&&\bm{i}&&&\\
&&&&&\bm{j} \: \bm{i}&&\\
&&&&&&\bm{j}&\\
&&&&&&&\bm{i}\: \bm{j}\\
\end{array}
\right)_{8 \times 12}
\]
\[
\bm{B}_1\bm{F}_2^T=\left(
\begin{array}{cccccccc}
\bm{i}\bm{i}^T&\bm{j}\bm{j}^T&&&&&&\bm{j}\bm{j}^T\\
&\bm{i}\bm{i}^T&\bm{j}\bm{j}^T&\bm{i}\bm{i}^T&&&&\\
&&&\bm{j}\bm{j}^T&\bm{i}\bm{i}^T&\bm{j}\bm{j}^T&&\\
&&&&&\bm{i}\bm{i}^T&\bm{j}\bm{j}^T&\bm{i}\bm{i}^T\\
\end{array}
\right)_{4\times 8}
\]
以此类推,$\bm{B}_k,k \geq 1$的结构应该是4k行乘以12k列的矩阵,每行有3个非零元,每列有一个非零元,而且$\bm{i},\bm{j}$交替出现。$\bm{F}_k$的结构应该是4k行乘以12(k-1)列的矩阵,每行有1个或2个非零元,非零元的数量交替出现,每列有一个非零元,而且$\bm{i},\bm{j}$交替出现。$\bm{B}_{k-1}\bm{F}_k^T$是4(k-1)行4k列的矩阵,每行有3个非零元,每列有1个或2个非零元,非零元的数量交替出现,非零元只能是$\bm{i}\bm{i}^T$或$\bm{j}\bm{j}^T$。$\bm{F}_k\bm{F}_k^T$是对角阵,对角元为$\bm{I}_i,\bm{I},\bm{I}_j,\bm{I},\bm{I}_i...$,
而$\bm{B_k}\bm{B_k}^T+\bm{F_k}\bm{F_k}^T$


在杨思怡的论文中将锚点的贡献简化为$x\bm{I}$的矩阵,x可以衡量贡献的大小,
在此基础上结合镜像后的FIM,我们先讨论第一步迭代计算:
\begin{equation}\label{eq:F1}
L_{2,2}^2=A_2-D_1^TA_1^{-1}D_1
\end{equation}•
这里$A_2=(x+2)\bm{I}$,是4(k-1)维的方阵,
$D_1=-\bm{F}_k\bm{B}_{k-1}^T$,是4k行4(k-1)列的矩阵,而且矩阵的每行相互正交,这反应了连接各个节点的链路集之间的局部独立性。因此$D_1^TD_1$是对角阵,等于$I+\bm{\epsilon}_{k-1}$

$A_1=(x+1)\bm{I}-\bm{\epsilon}_k$,是4k维矩阵,其中
$\bm{\epsilon}_k$只有主对角线上第1,k+1,2k+1,3k+1位置非零,分别为$\bm{I_i},\bm{I_j},\bm{I_i},\bm{I_j}$,在近似计算中,我们将首先考虑忽略$\bm{\epsilon}_k$对计算结果的影响。
代入计算(\ref{eq:F1})得
\begin{equation}
(x+2)\bm{I}-D_1^T((x+1)I-\bm{\epsilon}_k)^{-1}D_1
\end{equation}
忽略$\bm{\epsilon}_k$化简得
\begin{equation}
L_{2,2}^2 \approx ((x+2)-\frac{1}{x+1})\bm{I}-\frac{\bm{\epsilon_{k-1}}}{x+1}
\end{equation}•
进一步将$L_{2,2}^2$代入$A_3-D_2^TL^{-2}_{2,2}D_2$,考虑到对所有的中间$A_r=(x+2)\bm{I}$,在进一步忽略
$\bm{\epsilon}_{k-1}$后可以得到两步迭代的结果为:
\begin{equation}
L_{3,3}^2=((x+2)-\frac{1}{((x+2)-1/(x+1))})\bm{I}
\end{equation}
注意到$L_{3,3}^2$的维数已经降低到4(k-2),这样一直近似下去可以将$L_{k,k}^2$写成连分式的形式。
在最后一次计算中$A_{k+1}$实际上是$\bm{\Sigma}_0+\bm{B}_0\bm{B}_0^T=(x+2)\bm{I}$,为2乘2的矩阵,如果最后一次不做近似$D_{k+1}^TD_{k+1}=2\bm{I}$,从而得到
\begin{equation}
L_{k+1,k+1}^2=((x+2)-\frac{2}{x+2-\frac{1}{x+2-...\frac{1}{x+2-1/(x+1)}}})\bm{I}
\end{equation}
这是杨思怡的定理2.10的结果,从上面的推导可以看出,这个结果是在忽略了$\bm{\epsilon}_r,r=2,3,...k$的情况下推导出来的,直观上x比较大时近似程度很好(x很大时后面的分式都可以忽略掉),k步迭代从FIM的Chlosky Decomposition的左上角走到右下角,其中前k-1步做了近似,只有最后一步按准确解求的,因此分子会出现2.
\end{document}