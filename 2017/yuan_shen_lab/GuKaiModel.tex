\documentclass[12pt]{ctexart}
\usepackage{graphicx}
\usepackage{indentfirst}
\usepackage[a4paper, inner=1.5cm, outer=3cm, top=2cm, bottom=3cm, bindingoffset=1cm]{geometry}
\usepackage{epstopdf}
\usepackage{listings}
\usepackage{array}
\usepackage{soul}
\usepackage{fontspec}
\usepackage{bm}
\usepackage{gensymb}
\usepackage{todonotes}
\usepackage{amsmath, amsthm, amssymb}
\usepackage[citecolor=blue]{hyperref}
\newtheorem{definition}{Definition}
\newtheorem{thm}{Theorem}[section]
\newtheorem{cor}[thm]{Corollary}
\newtheorem{lem}[thm]{Lemma}
\DeclareMathOperator{\sgn}{sgn}
\theoremstyle{remark}
\newtheorem*{rem}{Remark}
\usepackage{makecell}
\usepackage[lofdepth,lotdepth]{subfig}
\setlength{\extrarowheight}{4pt}
\setlength{\parindent}{1cm}
\begin{document}
\title{\textbf{\fontsize{15.75pt}{\baselineskip}{讨论}}} 
\date{2017/3/1}
\author{\fontsize{12pt}{\baselineskip}{数33 赵丰}}
\maketitle
\large
已知一个节点的先验分布是二维高斯分布$(X,Y)$,X,Y方向的均值均为0,方差为$\sigma^2$,相关系数r=0,如果用极坐标表示,那么该节点到原点的距离为参数为$\sigma$的Rayleigh分布,而幅角为均匀分布。
考虑在x轴上较远的一点$(l_0,0)$上有一个观测站,求节点到观测站的距离的分布。

距离$L=\sqrt{(X-l_0)^2+Y^2},Y^2/\sigma^2$是自由度为1的卡方分布,$(X-L_0)^2$的分布由定义可求出为:
\begin{equation}
f(t)=\frac{1}{\sqrt{2\pi t\sigma^2}}exp(-\frac{t+l_0^2}{2\sigma^2})\cosh(\frac{l_0\sqrt{t}}{\sigma^2})
\end{equation}
注意$(X-L_0)^2$和$(X+L_0)^2$有相同的分布,由于反函数不存在,不可以直接对pdf做变量替换求出上式。
因为$(X-l_0)^2$与$Y^2$独立,所以$L^2$的pdf为$(X-l_0)^2$与$Y^2$pdf的卷积。
化简有:
\begin{equation}
f_{L^2}(y)=\frac{exp(-\frac{y+l_0^2}{2\sigma^2})}{2\pi\sigma^2}\int_0^y \frac{\cosh(\frac{l_0\sqrt{t}}{\sigma^2})}{\sqrt{t}\sqrt{y-t}}dt
\end{equation}
对上式做$u=\sqrt{t/y}$的变量替换:
\begin{equation}\label{eq:Int}
f_{L^2}(y)=\frac{exp(-\frac{y+l_0^2}{2\sigma^2})}{\pi\sigma^2}\int_0^1 \frac{\cosh(\frac{l_0\sqrt{y}u}{\sigma^2})}{\sqrt{1-u^2}}du
\end{equation}
设$k=\frac{l_0\sqrt{y}}{\sigma^2}$因为
\begin{equation}
\cosh(ku)=\sum_{n=0}^{+\infty} \frac{(ku)^{2n}}{(2n)!}
\end{equation}
代入(\ref{eq:Int}),注意到:
\begin{equation}
\int_0^1 \frac{u^{2n}}{\sqrt{1-u^2}}du=\int_0^{\frac{\pi}{2}} \sin^{2n} \theta d\theta=\frac{\pi}{2}\frac{(2n-1)!!}{(2n)!!} 
\end{equation}
因此
\begin{equation}
f_{L^2}(y)=\frac{exp(-\frac{y+l_0^2}{2\sigma^2})}{2\sigma^2}\sum_{n=0}^{+\infty}\frac{(l_0^2 y/\sigma^2)^n}{((2n)!!)^2}
\end{equation}
进一步求出
\begin{equation}
f_{L}(y)=\frac{yexp(-\frac{y^2+l_0^2}{2\sigma^2})}{\sigma^2}\sum_{n=0}^{+\infty}\frac{(l_0^2 y^2/\sigma^2)^n}{((2n)!!)^2}
\end{equation}
与$R=\sqrt{X^2+Y^2}$的分布函数为$f_{R}(y)=\frac{y}{\sigma^2}exp(-\frac{y^2}{2\sigma^2})$进行对比
\end{document}
