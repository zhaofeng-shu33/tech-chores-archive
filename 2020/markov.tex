\documentclass{article}
\usepackage{amsmath}
\usepackage{url}
\title{probability-theory}
\author{zhaof17 }
\date{December 2020}

\begin{document}

\maketitle
\section{Problem}
Let $\{X_n\}_{n=0}^{\infty}$
be a Markov chain with state space
$\{0,1,2,\dots\}$.
The transition probabilities are
$p_{0,1}=1$,
$p_{2n-1, 2n+1} = p$,
$p_{2n-1, 2n} = 1-p$,
$p_{2n, 2n+1} = p$,
$p_{2n, 2n-2} = 1-p$,
for all $n\geq 1$ where $p\in (0,1)$ is some constant.
For every $p\in (0,1)$,
indicate whether the Markov chain is transient,
null recurrent, or positive recurrent.
Prove your conclusion. When the Markov chain is
positive recurrent, calculate the stationary
distribution.
\section{Preliminary result}
The definition of transient, null recurrent and positive
recurrent chain can be found in \cite{def}.

If the Markov chain is positive recurrent. Using
$x = x P$ we can get $x_0 = c$ and

\begin{align*}
x_{2n} &= \frac{p^{n-1} c (2-p)^{n-1}}{(1-p)^n} (n\geq 1) \\
x_{2n+1} & = \frac{p^n c (2-p)^n}{(1-p)^n} (n\geq 0)
\end{align*}
This result can be proved using induction ($x=xP$) on:
\begin{align*}
    (1-p)(x_{2n-1} + x_{2n+2}) &= x_{2n} \\
    p(x_{2n-1} + x_{2n}) &= x_{2n+1}
\end{align*} for $n\geq 1$
while the initial condition is
$x_0=x_1=c, x_2(1-p)=x_0$.
This method is used in Example 15, Chapter 8 of \cite{prob}.
Although the transition matrix $P$ is infinite, we can
still use the steady state equation.

$c$ is the normalization constant such that
$\sum_{n=0}^{+\infty} x_n = 1$.
The series converges if $\frac{(p)(2-p)}{1-p} < 1$,
from which we can get the necessary condition on $p$ for
the Markov chain to be positive-recurrent:
$p > \frac{3-\sqrt{5}}{2}$.
We further guess that the chain is null-recurrent
when $p=\frac{3-\sqrt{5}}{2}$
and transient when $p<\frac{3-\sqrt{5}}{2}$.
\begin{thebibliography}{9}
\bibitem{def} \url{https://en.wikipedia.org/wiki/Markov_chain#Properties}
\bibitem{prob} Chung, Kai Lai. Elementary probability theory with stochastic processes. Springer Science \& Business Media, 2003 Fourth Edition.
\end{thebibliography}
\end{document}
